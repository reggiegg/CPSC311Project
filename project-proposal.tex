% This is "sig-alternate.tex" V2.0 May 2012
% This file should be compiled with V2.5 of "sig-alternate.cls" May 2012
%
% This example file demonstrates the use of the 'sig-alternate.cls'
% V2.5 LaTeX2e document class file. It is for those submitting
% articles to ACM Conference Proceedings WHO DO NOT WISH TO
% STRICTLY ADHERE TO THE SIGS (PUBS-BOARD-ENDORSED) STYLE.
% The 'sig-alternate.cls' file will produce a similar-looking,
% albeit, 'tighter' paper resulting in, invariably, fewer pages.
%
% ----------------------------------------------------------------------------------------------------------------
% This .tex file (and associated .cls V2.5) produces:
%       1) The Permission Statement
%       2) The Conference (location) Info information
%       3) The Copyright Line with ACM data
%       4) NO page numbers
%
% as against the acm_proc_article-sp.cls file which
% DOES NOT produce 1) thru' 3) above.
%
% Using 'sig-alternate.cls' you have control, however, from within
% the source .tex file, over both the CopyrightYear
% (defaulted to 200X) and the ACM Copyright Data
% (defaulted to X-XXXXX-XX-X/XX/XX).
% e.g.
% \CopyrightYear{2007} will cause 2007 to appear in the copyright line.
% \crdata{0-12345-67-8/90/12} will cause 0-12345-67-8/90/12 to appear in the copyright line.
%
% ---------------------------------------------------------------------------------------------------------------
% This .tex source is an example which *does* use
% the .bib file (from which the .bbl file % is produced).
% REMEMBER HOWEVER: After having produced the .bbl file,
% and prior to final submission, you *NEED* to 'insert'
% your .bbl file into your source .tex file so as to provide
% ONE 'self-contained' source file.
%
% ================= IF YOU HAVE QUESTIONS =======================
% Questions regarding the SIGS styles, SIGS policies and
% procedures, Conferences etc. should be sent to
% Adrienne Griscti (griscti@acm.org)
%
% Technical questions _only_ to
% Gerald Murray (murray@hq.acm.org)
% ===============================================================
%
% For tracking purposes - this is V2.0 - May 2012

\documentclass{sig-alternate}
\usepackage{hyperref}
\usepackage{breakurl}


\makeatletter
\def\@copyrightspace{\relax}
\makeatother

\begin{document} \sloppy

%
% --- Author Metadata here ---
% \conferenceinfo{University of British Columbia}{Vancouver, BC, Canada}
% \CopyrightYear{2014} % Allows default copyright year (20XX) to be over-ridden - IF NEED BE.
% \crdata{}  % Allows default copyright data (0-89791-88-6/97/05) to be over-ridden - IF NEED BE.
% --- End of Author Metadata ---

\title{PROPOSAL: A Stack Game in a Stack Language}
% \subtitle{[Extended Abstract]
% \titlenote{A full version of this paper is available as
% \textit{Author's Guide to Preparing ACM SIG Proceedings Using
% \LaTeX$2_\epsilon$\ and BibTeX} at
% \texttt{www.acm.org/eaddress.htm}}}
%
% You need the command \numberofauthors to handle the 'placement
% and alignment' of the authors beneath the title.
%
% For aesthetic reasons, we recommend 'three authors at a time'
% i.e. three 'name/affiliation blocks' be placed beneath the title.
%
% NOTE: You are NOT restricted in how many 'rows' of
% "name/affiliations" may appear. We just ask that you restrict
% the number of 'columns' to three.
%
% Because of the available 'opening page real-estate'
% we ask you to refrain from putting more than six authors
% (two rows with three columns) beneath the article title.
% More than six makes the first-page appear very cluttered indeed.
%
% Use the \alignauthor commands to handle the names
% and affiliations for an 'aesthetic maximum' of six authors.
% Add names, affiliations, addresses for
% the seventh etc. author(s) as the argument for the
% \additionalauthors command.
% These 'additional authors' will be output/set for you
% without further effort on your part as the last section in
% the body of your article BEFORE References or any Appendices.

\numberofauthors{5} %  in this sample file, there are a *total*
% of EIGHT authors. SIX appear on the 'first-page' (for formatting
% reasons) and the remaining two appear in the \additionalauthors section.
%
\author{
% You can go ahead and credit any number of authors here,
% e.g. one 'row of three' or two rows (consisting of one row of three
% and a second row of one, two or three).
%
% The command \alignauthor (no curly braces needed) should
% precede each author name, affiliation/snail-mail address and
% e-mail address. Additionally, tag each line of
% affiliation/address with \affaddr, and tag the
% e-mail address with \email.
%
% 1st. author
\alignauthor
Reggie Gillett\\
\affaddr{27133123}\\
\affaddr{w6k8}\\
\email{reggie.gillett@gmail.com}
% 2nd. author
\alignauthor
Graham St-Laurent\\
\affaddr{23310121}\\
\affaddr{i5l8}\\
\email{gstlaurent@gmail.com}
% 3rd. author
\alignauthor Lynsey Haynes\\
\affaddr{12686119}\\
\affaddr{a4h8}\\
\email{lynseyahaynes@gmail.com}
\and  % use '\and' if you need 'another row' of author names
% 4th. author
\alignauthor Brittany Roesch\\
\affaddr{22015119}\\
\affaddr{r1d8}\\
\email{brroesch@gmail.com}
% 5th. author
\alignauthor Gord Minaker\\
\affaddr{34615112}\\
\affaddr{u4s8}\\
\email{gordonminaker@hotmail.com}
}
% % There's nothing stopping you putting the seventh, eighth, etc.
% % author on the opening page (as the 'third row') but we ask,
% % for aesthetic reasons that you place these 'additional authors'
% % in the \additional authors block, viz.
% \additionalauthors{Additional authors: John Smith (The Th{\o}rv{\"a}ld Group,
% email: {\texttt{jsmith@affiliation.org}}) and Julius P.~Kumquat
% (The Kumquat Consortium, email: {\texttt{jpkumquat@consortium.net}}).}
% \date{30 July 1999}
% Just remember to make sure that the TOTAL number of authors
% is the number that will appear on the first page PLUS the
% number that will appear in the \additionalauthors section.

\maketitle
\begin{abstract}
We are creating a substantial program in Factor; a programming language that none of us have studied previously. Factor is a concatenative, stack-based programming language and we will be using it to make a game which helps to teach users about stack-based languages.
\end{abstract}

% % A category with the (minimum) three required fields
% \category{H.4}{Information Systems Applications}{Miscellaneous}
% %A category including the fourth, optional field follows...
\category{D.3.3}{Programming Languages}{Language Constructs and Features}

\terms{Languages}

\keywords{Factor, games, stack-based languages, concatenative languages}

\section{Introduction}
We will be creating a substantial program in Factor - a stack-based
programming language that none of us have studied previously. 
Our game will educate and illustrate to players the
unique paradigm of concatenative languages. We intend to create
said game in Factor, which is itself a concatenative language.
This is not an accident. 

\section{The Game} \subsection{How it Meets Project Requirements}
\subsubsection*{\textit{Substantialness of the Program}}
We will be
creating a fully featured interactive game. This will be a substantial
undertaking as we intend to include several levels and some sort of
graphical user interface including a visual representation of a stack.
Concepts pertaining to stack-based and concatenative languages will be
illustrated through analogous gameplay. We have discussed the idea of
using the analogy of creating recipes to prepare various forms of
stack-like food (think: hamburgers, sandwiches, pancakes). Through
this parallel, players will manipulate ``recipes'' which are
representative of a \textit{concatenative} sequence of operation which
manipulate a stack. A player will be presented with a type of food
that they are required to prepare. They will be given a vocabulary of
actions from which they can construct recipes. These actions are a
direct analog to words as they are presented in Factor. When recipes
are executed, the user will be presented with their final product,
which will be compared to the desired outcome. If they match the
player progresses to the next level. Progression through levels will
present new foods which require increasingly complex recipes to
create. Each level will highlight some aspect particular to the
language, which highlights various aspects of our language of inquiry.
As it gets more complex, the analogy starts to break down, how does
one represent control flow words such as conditionals or loops in a
recipe for food? In order to surmount this obstacle we have discussed
having our levels become increasingly surreal.  Perhaps the player
will have to prepare a hamburger for an mystery diner, and will
include a condition that if the diner reveals himself to be a
vegetarian a veggie burger will be prepared otherwise a standard beef
burger will be prepared. While this example does not necessarily
illustrate the surreal quality we might incorporate, it does highlight
the more abstract challenges we will have to navigate.

\subsubsection*{\textit{Useful But Esoteric Nature of Factor and its Value}}
Factor is both a concatenative and stack-based language, but
unlike most languages following those paradigms, it is rich with
features that are found in more typical modern programming languages
(E.g., graphical, networking, threading, and I/O capabilities). This
leaves open the possibility of creating a substantial and useful
program without too much effort. Factor also implements a diverse
collection of functional patterns, making it easy to intuitively
manipulate sequential data. But unlike most high-level languages --
functional, or otherwise -- Factor provides the option of using low-level
 features such as manual memory management, pointer arithmetic,
and inline assembly code. Although we don't expect that coding our
program will take us in that particular direction, we find the
juxtaposition of high and low levels to be particularly unique in
Factor, and something that, under the right circumstances, could be
quite useful (albeit, potentially quite dangerous, too!). Ultimately,
Factor supports a diverse array of programming styles, thus leading to
diverse ways to solve problems, thus perhaps leading to surprising
solutions, and some of those might turn out to be quite enlightening.\\\\
In spite of the diverse approaches to programming that Factor
provides, its syntax demands that actual coding will always be stack-based
 and concatenative. It is the value of this syntax and its unique
patterns, that we find the most interesting, and potentially, the most
useful. Concatenative syntax is point-free (although Factor does
provide identifiers and its creators recommend using them for
particularly complex situations), and stack-based languages use
postfix operators; both of these syntactical styles are atypical among
today's most popular programming languages (with the exception of
Haskell, which has a popular point-free style). This means that Factor
has a unique way of expressing and exploring, and we will develop a
new way of looking at problems that could come in handy in our future
endeavours. In the course of working on our program, we will try to
use as many of these strange idioms and patterns as we can. The
creators of Factor have given evocative names to many of these; we
intend to produce a program turgid (where appropriate, that is) with
things like  \textit{Fried Quotations}, some of the numerous
\textit{combinators} (e.g., \textit{Cleave, Spread, Sequence, Curried
dataflow}), and, perhaps most interestingly, functions that assemble
or disassemble \textit{quotations} (lambdas), thus providing an
idiomatic way to ``write code that writes code'' [1]

\subsection{Our Approach}
\subsubsection*{\textit{Background Research
Report}}To complete this milestone, we will research even more
extensively about the advantages (and disadvantages) of Factor as a
language, concatenative languages in general, and how the
implementation of our game will relate to these features. We plan to
research about previous implementations of concatenative languages,
their use in industry, and previous applications of Factor. In
addition, we will layout the foundation for how our game will look,
what the user-interface will be, and how we plan to teach important
components of concatenative languages to the user.\\\\ After researching
the advantages of Factor, we will discuss how these features can be
implemented and used in our game. We will experiment to see how they
work in sample programs to better understand our knowledge of Factor.

\subsubsection*{\textit{Project Proof-of-Concept and Plan}}In this
stage we will firm up our plan and timeline to complete our super-
awesome (hamburger / sandwich / pancake) making game that teaches
users about concatenative languages without them even knowing it.
Ethical? maybe. Awesome? definitely. We will provide details about our
collaboration plan - when, where, and who will meet, the
implementation of the game - what it will look like and how it will
work, technical requirements - what is required for us to work
collaboratively and program in Factor, and give ourselves milestones
to be able to complete the implementation stress-free. Completing this
stage will mean we have reached our 80\% milestone.

\subsubsection*{\textit{Poster}} Our poster will include stacks on
stacks of stacks, hamburgers, and Factor-ial information. We will
present the benefits of Factor as a language, concatenative
languages, and build hype about how cool our game is so that we can
release it, sell it, and all graduate debt-free! We plan to use a
sturdy display board so that it can survive a small natural disaster.
We will have completed some work on the game by this point, so
hopefully we can have a demonstration for our peers to see.

\subsubsection*{\textit{Final Project}} In this stage, we will complete
the implementation of our game and compile our final report which
will explain the game and any information required to play it. We
will outline what we accomplished, how we met the goals in our
project plan, and put together our final report for handin. Competing
this stage will mean we have reached our 100\% milestone.

\subsection{Starting Points}
We have found several scholarly articles and online resources
which will aid in the development of our project:

\begin{itemize}
\item Pestov, S., Ehrenberg, D., Groff, J.: \textit{Factor: a dynamic stack-based programming
language.} In: DLS 2010 Proceedings of the 6th Symposium on Dynamic Languages
(2010)\\\\A paper which presents Factor as a language written by its creators. (Note the
reference to our own Gregor Kiczales in the bibliography)
\item Herzburg, D., Reichert, T.: \textit{Concatenative Programming:
An Overlooked Paradigm in Functional Programming} In: Proceedings of the 4th
international conference on software and data technologies (2009)
\\\\A paper which espouses the value of concatenative languages in software engineering 
research. The authors present a language they have developed called Concat which exemplifies the unique qualities of the paradigm.
\item \url{http://concatenative.org}\\
in particular:\\\url{http://concatenative.org/wiki/view/Factor}\\\\Concatenative.org, a wiki dedicated to concatenative languages.
\item Slava Pestov (October 27, 2008). \href{http://youtu.be/f_0QlhYlS8g}{Factor: An Extensible Interactive Language} (flv) (Tech talk). Google.\\\\A talk which describes the rational for Factor's creation, an overview of the language and its features, and a demonstration of how it can be used.
\item Zed Shaw (2008). \href{http://vimeo.com/2723800}{The ACL is Dead} (flv) (CUSEC 2008). CUSEC.\\\\
A presentation written in Factor which mentions and praises Factor
\item The creators of Factor have also provided numerous resources on their webpage:\\\\\url{http://factorcode.org/}
\end{itemize}


\section{Summary} We are going to be researching Factor's advantages
as an exotic programming language to incorporate them into a game. The
focus will be on understanding and eventually implementing Factor's
unique features such as its stack-based nature, dynamic typing, point-
free style, and multiple argument returns. This will allow the user to
visually examine stack-based languages. Overall, we will evaluate
Factor's usefulness as a programming language to develop an
interactive game.

\section*{References}
[1] Pestov, Slava. Control Flow Cookbook - Factor Documentation. 
\burl{http://docs.factorcode.org/content/article-cookbook-combinators.html}(2014).



\balancecolumns
\end{document}
